%%MaD.tex - Notes taken for Materials and Devices Lecture
%%Author: Andy Goetz
%%Date Modified: 10-7-09
%%License: Ask me before reproducing/modifying, etc.


\documentclass{article}

%Make sure you have the file ShumanNote.scy in the same directory as
%this one. It has contains the style sheet for ECE111, and is needed
%to standardize the layout of LateX documents created for the class.
\usepackage{ShumanNotes} 
\usepackage{tikz}
\usepackage{program}
\usepackage{listings}
\pdfpagewidth 8.5in 
\pdfpageheight 11in

%This package is used to line up pictures 
\usepackage{graphicx}
\usepackage{fancyvrb}
\usepackage{listings}
%allows cursive font
%\usepackage{amsmath}

%allows hyperlinks 
\usepackage{hyperref}

\newcommand{\HRule}{\rule{\linewidth}{0.5mm}} 

\lhead{Source Code}

\begin{document}

%% These commands allow me to use cursive letter for things such as
%% length.  Note that on ubuntu linux, this required installation of
%% the package 'texlive-fonts-extra'. 
%% Taken from
%% http://www.latex-community.org/forum/viewtopic.php?f=5&t=1404&start=0
\newenvironment{frcseries}{\fontfamily{frc}\selectfont}{}
\newcommand{\textfrc}[1]{{\frcseries#1}}
\newcommand{\mathfrc}[1]{\text{\textfrc{#1}}}

\section{pseudocode.txt}
\lstinputlisting{pseudocode.txt}
\newpage

\section{memtest.v}
\lstinputlisting[language=Verilog]{../BEEFA55/Source/memtest.v}
\newpage 

\section{PROJECT.v}
\lstinputlisting[language=Verilog]{../BEEFA55/Source/PROJECT.v}
\newpage 

\section{INS\_CACHE.v}
\lstinputlisting[language=Verilog]{../BEEFA55/Source/INS_CACHE.v}
\newpage 

\section{DATA\_CACHE.v}
\lstinputlisting[language=Verilog]{../BEEFA55/Source/DATA_CACHE.v}
\newpage 

\section{STATS.v}
\lstinputlisting[language=Verilog]{../BEEFA55/Source/STATS.v}
\newpage 


\section{testplan}
\lstinputlisting{../vectors/testplan}
\newpage 

\section{Testbench Output}
\lstinputlisting{../vectors/testresults}
\end{document}
